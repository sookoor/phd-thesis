\usepackage{amsmath}
\usepackage{amssymb}
\usepackage{amsthm}
\usepackage{fancyvrb}
\usepackage[tight]{subfigure}
\usepackage{multirow}
\usepackage[english]{babel}
\usepackage[usenames]{color}% for color
\usepackage{comment}
\usepackage{flushend}
\usepackage{framed}
\usepackage{atbeginend}
\usepackage{paralist} % for concise lists
\usepackage{verbatim} % for comment blocks
\usepackage{wrapfig} % for within-text-flow wrap-around figures
\usepackage{listings}
\usepackage{algorithmic}
\usepackage{alltt}
\usepackage{fancyhdr}
\usepackage{multicol}
\usepackage{ccaption}

%\usepackage[nolineno]{lgrind}

% ???
%\LGnorulestrue

% If you're using lgrind for C code
%\newif\ifusegrind
%\usegrindfalse

% For a black square at the end of proofs
\renewcommand{\qedsymbol}{$\blacksquare$}

% Theorem-like environments with title Theorem and title Lemma, having
% chapter-relative numbering (i.e. first theorem in Ch. 5 is 5.1).
\newtheorem{theorem}{Theorem}[chapter]
\newtheorem{lemma}{Lemma}[chapter]

% For nicely formatted URLs
\usepackage{url}

% I had a conflict with \url in BiBTeX, so here's an alias. The BiBTeX
% file has this sort of thing:
%   note = "URL: {\urlBiBTeX{http://coppit.org/}}",
\newcommand{\urlBiBTeX}[1]{\url{#1}}

% xspace is used in macros to add a space unless the macro is followed
% by certain punctuation characters
%% \iftth
%% \newcommand{\xspace}{\ }
%% \else
%% \usepackage{xspace}
%% \fi

% Use tex4ht if ht is true
%% \ifht
%% % ",2" is causing:
%% % ! LaTeX Error: Option clash for package tex4ht.
%%   \usepackage[html,2]{tex4ht}
%% %  \usepackage[html]{tex4ht}
%% \else
%% \fi

% \ifwww can be used in the document to tweak it for HTML output
\newif\ifwww

%% \iftth
%% \wwwtrue
%% \fi

%% \ifht
%% \wwwtrue
%% \fi

% Change [1][2][3] to [1,2,3]
\usepackage{cite}

% Generate links in html and PDF.
%\usepackage[breaklinks=true,letterpaper=true,a4paper=false]{hyperref}

% Tell LaTeX to not "bottom justify" text. This prevents ugly
% spaces between paragraphs in columns when LaTeX stretches them.
\raggedbottom

% Help LaTeX not violate the column margins
\tolerance=50000

% Prevent widows and orphans (lines all by themselves at the top &
% bottom of pages)
\widowpenalty=1500
\clubpenalty=1500

% ???
%\relpenalty10000
%\binoppenalty10000

%% \ifdraft
%%   \pagestyle{myheadings} \markright{Draft \today: Please do not redistribute.}
%% \else
%%   \pagestyle{headings}
%% \fi

% pdflatex stuff
\newif\ifpdf
\ifx\pdfoutput\undefined
        \pdffalse
\else
        \pdftrue
\fi

% Tell graphicx to slurp in PDF or EPS figures depending on whether
% we're processing using pdflatex or not
%% \ifpdf
%%         \usepackage[final,pdftex]{graphicx}
%%         \pdfcompresslevel=9
%% 	\DeclareGraphicsExtensions{.pdf}

%%   % Graphics are in figures directory
%% \else
%%         \usepackage[final]{graphicx}
%% 	\DeclareGraphicsExtensions{.eps}

%%   % Graphics are in figures directory
%% \fi
\graphicspath{{fig}}

\usepackage{rotating} % for rotated table captions (currently) - Pieter

% Set margins for one- or two-sided printing
%% \iftwoside
%%   \evensidemargin0in
%%   \oddsidemargin0.46875in
%%   \textwidth5.97in
%% \else
%%   \evensidemargin0.46875in
%%   \oddsidemargin0.46875in
%%   \textwidth5.98in
%%   %\textwidth6.05in
%% \fi

% Extra formatting stuff
%\setlength{\textheight}{8.75in}
%\setlength{\textwidth}{6.8in}
%\setlength{\topmargin}{0.25in}
%\setlength{\headheight}{0.0in}
%\setlength{\headsep}{0.0in}
%\setlength{\oddsidemargin}{-.19in}
%\setlength{\parindent}{1pc}

%\setlength{\footskip}{35pt}

\setlength{\topmargin}{0in}
\setlength{\textheight}{8.5in}

% A code environment for putting code in figures
\DefineVerbatimEnvironment{figurecodeverbatim}%
  {Verbatim}%
  {fontfamily=tt,%
   fontsize=\small,%
   commandchars=\\\{\},%
   formatcom=\def\{{\symbol{123}}\def\}{\symbol{125}}\def\\{\symbol{92}},%
   listparameters=\setlength{\topsep}{0pt}%
                  \setlength{\partopsep}{0pt}%
                  \setlength{\parskip}{0pt}%
   }

% A code environment for putting code in text
\DefineVerbatimEnvironment{quotecodeverbatim}%
  {Verbatim}%
  {fontfamily=tt,%
   fontsize=\small,%
   commandchars=\\\{\},%
   formatcom=\def\{{\symbol{123}}\def\}{\symbol{125}}\def\\{\symbol{92}}%
   }

%%\newcommand{\fancyfloatrule}{\ifwww\else{\noindent\hrulefill\par}\fi}
\newcommand{\fancyfloatsize}{\small}

%%\newcommand{\boldheading}[1]{{\vspace{0.1in}\noindent \bf #1} \hspace{0.06in}}

% Puts a line above and below a figure. This is nice for text-only
% figures.
\newenvironment{fancyfigure}[1][tbp]%
  {\begin{figure}[#1]%
   \fancyfloatsize%
   \fancyfloatrule%
  }
  {\fancyfloatrule%
   \end{figure}}

% Month in text format
\newcommand*{\Month}{%
  \ifcase\month \or
  January\or February\or March\or April\or May\or June\or
  July\or August\or September\or October\or November\or
  December\fi \xspace
}

% Year
\newcommand*{\Year}{\number\year\xspace}

% Needed for Ventry to work
\usepackage{calc}

% To get definitions that line up with the longest term.
\newenvironment{Ventry}[1]%
 {\begin{list}{}{\renewcommand{\makelabel}[1]{##1:\hfil}%
  \settowidth{\labelwidth}{\textsf{#1:}}%
  \setlength{\itemsep}{0pt}%
  \setlength{\parsep}{0pt}%
  \setlength{\leftmargin}{\labelwidth+\labelsep}}}%
  {\end{list}}

\definecolor{dkgreen}{rgb}{0,0.6,0} 
\definecolor{gray}{rgb}{0.5,0.5,0.5}  
\definecolor{gray_ulisses}{gray}{0.55}
\definecolor{castanho_ulisses}{rgb}{0.71,0.33,0.14}
\definecolor{preto_ulisses}{rgb}{0.41,0.20,0.04}
\definecolor{green_ulises}{rgb}{0.2,0.75,0}
\def\PAIR#1#2{\langle#1, #2\rangle} 
%%\newcommand\aand{\ \wedge\ } 
%%\newcommand\oor{\ \vee\ }    
\lstdefinelanguage{MacroLab}
{
  basicstyle=\ttfamily\small,
  sensitive=true,
  morecomment=[l][\color{red}\small]{\%},
  morestring=[b]',
  stringstyle=\color{dkgreen}, %% string color   
  showstringspaces=false,   
  numbers=left, %% number lines    
  numberstyle=\ttfamily\small\color{gray}, %% the style of the number
  numberblanklines=true,
  showspaces=false,
  showtabs=false,
  xleftmargin=5pt,
  xrightmargin=-20pt,
  stepnumber=1,   
  numbersep=5pt,   
  backgroundcolor=\color{white},   
  tabsize=2,
  captionpos=b,   
  breaklines=true,
  breakatwhitespace=false,
  keywordstyle=\color{blue},   
  ndkeywordstyle=\color{red},   
  breakautoindent=true,
  morekeywords={if,while,RunTimeSystem,newSensorVector,newMacrovector,every,sense,end,neighborReflection,find,sum,max,getTime,
  getNodes}
%%   morekeywords={while,if,end,numel,find,max,RunTimeSystem,Macrovector,neighborReflection,CAMERAFOCUS,DISPLAY,createSensorVector,sense,getNodes,getTime,every,sum,createMacrovector,sread}
}

\lstnewenvironment{macrolab}
{\lstset{language=MacroLab}}
{}

\lstdefinelanguage{nesC}
{
  basicstyle=\ttfamily\small,
  sensitive=true,
  morecomment=[l][\color{red}\small]{\/\/},
  morestring=[b]',
  stringstyle=\color{dkgreen}, %% string color   
  showstringspaces=false,   
  numbers=left, %% number lines    
  numberstyle=\ttfamily\small\color{gray}, %% the style of the number
  numberblanklines=true,
  showspaces=false,
  showtabs=false,
  xleftmargin=5pt,
  xrightmargin=-20pt,
  stepnumber=1,   
  numbersep=5pt,   
  backgroundcolor=\color{white},   
  tabsize=2,
  captionpos=b,   
  breaklines=true,
  breakatwhitespace=false,
  keywordstyle=\color{blue},   
  ndkeywordstyle=\color{red},   
  breakautoindent=true,
  morekeywords={if,else,while,and,or,not,abstract,as,atomic,async,call,command,components,configuration,default,event,implementation,interface,includes,module,norace,post,provides,signal,task,uses,result_t,SUCCESS,FAIL,TRUE,FALSE}
}

\lstnewenvironment{nesc}
{\lstset{language=nesC}}
{}

