\begin{abstract}
  Cyber-Physical Systems (CPSs) combine low-power radios with tiny embedded
  processors in order to simultaneously cover large geographic areas {\em and}
  provide high-resolution sensing/actuation. However, CPSs are extremely
  difficult to program. This issue is addressed by presenting a macroprogramming
  framework called {\em MacroLab} that offers a vector programming abstraction
  similar to Matlab for Cyber-Physical Systems.  The user writes a single
  program for the entire network using Matlab-like operations such as {\tt
  addition}, {\tt find}, and {\tt max}. The framework executes these operations
  across the network in a distributed fashion, a centralized fashion, or
  something between the two -- whichever is most efficient for the target
  deployment. This approach is called {\em deployment-specific code
  decomposition} (DSCD).  MacroLab programs can be executed on mote-class
  hardware such as the Telos motes. The results indicate that MacroLab
  introduces almost no additional overhead in terms of message cost, power
  consumption, memory footprint, or CPU cycles over TinyOS programs.

  As a crucial component of the application development cycle, debugging CPSs is
  addressed by {\em MDB}, the first system to support the debugging of
  macroprograms.  MDB allows the user to set breakpoints and step through a
  macroprogram using a source-level debugging interface similar to GDB, a
  process called {\em macrodebugging}.  A key challenge of MDB is to step
  through a macroprogram in sequential order even though it executes on the
  network in a distributed, asynchronous manner.  Besides allowing the user to
  view distributed state, MDB also provides the ability to search for bugs over
  the entire history of distributed states.  Finally, MDB allows the user to
  make hypothetical changes to a macroprogram and to see the effect on
  distributed state without the need to redeploy, execute, and test the new
  code. Macrodebugging is both easy and efficient: MDB consumes few system
  resources and requires few user commands to find the cause of bugs. A
  lightweight version of MDB, called {\em MDB Lite}, is also provided. It can be
  used during the deployment phase to reduce resource consumption while still
  eliminating the possibility of heisenbugs: changes in the manifestation of
  bugs caused by enabling or disabling the debugger.

  While a large number of CPS applications have been developed over the last few
  years, very few, if any, have been developed using a macroprogramming
  language. This is in spite of the vast array of macroprogramming languages,
  and abstractions, that are available. In an attempt to understand why
  macroprogramming languages are not being used for CPS application development
  and in order to evaluate the effectiveness of MacroLab in implementing a
  real-world CPS, a series of case studies involving the occupancy-based control
  of a Heating, Ventilation, and Air Conditioning (HVAC) system is
  implemented. Occupant-oriented HVAC control was selected due to it being a
  complete cyber-physical system involving the interaction between the real
  world and computation. Unlike many sensor network applications that involve
  primarily data collection, occupant-oriented HVAC control requires all three
  constituents of a complete CPS: sensing, computation, and actuation. The
  computation involved in these case studies range from making control decisions
  to predictions. 

  The case studies can be written in MacroLab in fewer lines of code than their
  implementations in Python and MacroLab can optimize their implementations
  depending on the topology of the network and capabilities of the sensors. Yet,
  the freedom for optimization decreases as the complexity of computation
  increases, or the size of the network decreases. Thus, for future
  cyber-physical systems the abstractions provided by macroprogramming systems,
  such as MacroLab presenting all sensor values as vectors and actuations as
  function calls, would be more beneficial than any implementation optimizations
  they may afford. 
\end{abstract}
