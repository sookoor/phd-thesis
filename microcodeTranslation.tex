\chapter{Code Translation}
\label{appendix:codeTranslation}

\begin{table}
\begin{tabular}{|l|l|l|}
\noalign{\hrule} 
 & & \multicolumn{1}{c|}{$\mathbf{lhs = M(a_1,\ldots,a_n) \ \% \ Synchronous\ Read }$ } \\
\noalign{\hrule}
\multirow{2}{*}{\begin{sideways}Cnt. or Ref.\end{sideways}} & R& %
\tt\begin{tabular}{l} %
owner\_id = RTS.owner(cur\_pc(), M); \\
RTS.notify(cur\_pc(), owner\_id); \\
\end{tabular} \\ \cline{2-3}
 & L & %
% Centralized, local (i.e. we have control)
\tt\begin{tabular}{l} %
node\_ids = source\_nodes(a1); \\
RTS.wait(cur\_pc(), node\_ids); \\
lhs = local( M(a1,...,an) ); \\
\end{tabular} \\ \noalign{\hrule}
\multirow{2}{*}{\begin{sideways}Distributed\end{sideways}} & R & %
\tt\begin{tabular}{l} %
if (a1 contains node\_id()) then \\
~~owner\_id = RTS.owner(cur\_pc(), M); \\
~~RTS.send(owner\_id, \\
~~~~local(M(node\_id(), a2,...,an) )) \\
~~RTS.notify(cur\_pc(), owner\_id); \\
fi; \\
\end{tabular} \\ \cline{2-3}
 & L &  %
\tt\begin{tabular}{l}%
node\_ids = source\_nodes(a1); \\
RTS.wait(cur\_pc(), node\_ids); \\
lhs = local( M(a1,...,an) ); 
\end{tabular}   \\
\noalign{\hrule}
\end{tabular}
\caption[Pseudocode for mote-level microcode translation of synchronous read] {For each data representation, the row
marked L (for \emph{local}) denotes the code for the mote that will
perform the operation (i.e., the locus of synchronization); R (for
\emph{remote}) marks the code for all other nodes. {\tt M} is a
macrovector; {\tt lhs} and {\tt rhs} are normal vectors. The
mote-local representation of {\tt x} is given by {\tt local(x)}. The
{\tt owner(PC,M)} function gives the ID of the node requesting
the read or write operations on macrovector {\tt M} at
location {\tt PC} in the macroprogram.}
\label{table:synchronousRead}
\end{table}

\begin{table}
\begin{tabular}{|l|l|l|}
\noalign{\hrule} 
 & & \multicolumn{1}{c|}{$\mathbf{M(a_1,\ldots,a_n) = rhs \ \% \ Synchronous\ Write}$ } \\
\noalign{\hrule}
\multirow{2}{*}{\begin{sideways}Centralized\end{sideways}} & R& %
\tt\begin{tabular}{l} %
if (a1 contains node\_id()) then\\
~~owner\_id = RTS.owner(cur\_pc(), M); \\
~~RTS.wait(cur\_pc(), owner\_id); \\
fi; \\
\end{tabular} \\ \cline{2-3}
 & L & %
\tt\begin{tabular}{l} %
node\_ids = source\_nodes(a1); \\
local( M(a1,...,an) ) = rhs; \\
RTS.notify(cur\_pc(), node\_ids); \\
\end{tabular} \\ \noalign{\hrule}
\multirow{2}{*}{\begin{sideways}Reflected\end{sideways}} & R & %
\tt\begin{tabular}{l} %
if (a1 contains node\_id()) then \\
~~owner\_id = owner(cur\_pc(), M); \\
~~RTS.receive(owner\_id, \\
~~~~local( M(a1,...,an) )); \\
~~RTS.wait(cur\_pc(), owner\_id); \\
fi; \\
\end{tabular} \\ \cline{2-3}
 & L & %
\tt\begin{tabular}{l} %
node\_ids = source\_nodes(a1); \\
local( M(a1,...,an) ) = rhs; \\
foreach (node\_id in node\_ids) do \\
~~RTS.send(node\_id, \\
~~~~~~~~~~~local( M(a1,...,an) )); \\
done; \\
RTS.notify(cur\_pc(), node\_ids); \\
\end{tabular} \\ \noalign{\hrule}
\multirow{2}{*}{\begin{sideways}Distributed\end{sideways}} & R & %
\tt\begin{tabular}{l} %
if (a1 contains node\_id()) then \\
~~owner\_id = RTS.owner(cur\_pc(), M); \\
~~RTS.receive(owner\_id, \\
~~~~local(M(node\_id(), a2,...,an))); \\
~~RTS.wait(cur\_pc(), owner\_id); \\
fi; \\
\end{tabular} \\ \cline{2-3}
 & L &  %
\tt\begin{tabular}{l}%
node\_ids = source\_nodes(a1); \\
local( M(a1,...,an) ) = rhs; \\
foreach (node\_id in node\_ids) do \\
~~RTS.send(node\_id,\\
~~~~~~~~~~~local( M(a1,...,an) )); \\
done; \\
RTS.notify(cur\_pc(), node\_ids); \\
\end{tabular}   \\
\noalign{\hrule}
\end{tabular}
\caption[Pseudocode for mote-level microcode translation of synchronous write] {For each data representation, the row
marked L (for \emph{local}) denotes the code for the mote that will
perform the operation (i.e., the locus of synchronization); R (for
\emph{remote}) marks the code for all other nodes. {\tt M} is a
macrovector; {\tt lhs} and {\tt rhs} are normal vectors. The
mote-local representation of {\tt x} is given by {\tt local(x)}. The
{\tt owner(PC,M)} function gives the ID of the node requesting
the read or write operations on macrovector {\tt M} at
location {\tt PC} in the macroprogram.}
\label{table:synchronousWrite}
\end{table}
