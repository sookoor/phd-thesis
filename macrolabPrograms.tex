\chapter{MacroLab Programs}
\label{appendix:macrolabPrograms}

\begin{figure}
  \begin{macrolab}
RTS = RunTimeSystem();
lSensors = SensorVector('lightSensor','uint16');
lightValues = Macrovector('uint16');
every(1000)
  lightValues =  sense(lSensors);
  BASE_DISPLAY(lightValues);
end
  \end{macrolab}
  \caption[A data collection application (Surge) in MacroLab]{Surge reads sensor
  values and displays them at the base station.  {\tt BASE\_DISPLAY} is
  implemented within the RTS and sends a message to a base station for display.}
  \label{code:Surge}
\end{figure}

\begin{figure}  
  \begin{macrolab}
motes = RTS.getMotes('type', 'tmote')
magSensors = SensorVector(motes, 'magnetometer');
magVals = Macrovector(motes);
neighborMag = neighborReflection(motes, magVals);
THRESH = uint8(500);
every(1000)
  magVals =  magSensors.sense();
  active = find(sum(neighborMag > THRESH, 2) > 3);
  maxNeighbor = max(neighborMag, 2);
  leaders = find(maxNeighbor(active) == magVal(active));
  CAMERAFOCUS(leaders);
end
  \end{macrolab}
  \caption[A tracking application (PEG) in MacroLab]{Every 1000\ms, nodes take a
  reading from their magnetometers and share the values with their neighbors.
  If more than three nodes in a neighborhood sense a magnetometer value about a
  threshold, a leader is elected from among them and a camera is focused on
  it. {\tt CAMERAFOCUS} is implemented within the RTS and sends a message to the
  camera, which is at the base station.}
  \label{code:PEG}
\end{figure}

\begin{figure*}[t]
  \begin{macrolab}
    RTS = RunTimeSystem();
    busstops = RTS.getNodes('stopnode'); 
    buses = RTS.getNodes('bus');
    estimates = Macrovector(busstops, length(buses) ,'uint16');
    arrivals = Macrovector(busstops, length(buses) ,'uint16'));
    travelTime = Macrovector(busstops, length(busstops), length(buses) ,'uint16'));
    busSensors = SensorVector('BusSensor',busstops,'uint16');
    routes = uint8({[1 2 3 4], [ 5 6 7 8]}); %Example routes

    while(1)
      [busID,r] =  sense(busSensors);
      busTime = RTS.getTime();
      travelTime(routes{r},routes{r},busID)[1,3] = busTime - arrivals(routes{r}, busID);
      arrivals(routes{r},busID)[1,2] = busTime;
      estimates(routes{r},busID) = travelTime(routes{r},routes{r},busID)[2,3] + busTime;
      BASE_DISPLAY(estimates(routes{r},:));
    end
  \end{macrolab}
  \caption[A bus tracking application in MacroLab]{MacroLab code for the bus
  tracking application.}
  \label{code:BusTracking}
\end{figure*}
