\chapter{Room-Level Zoning}
\label{appendix:roomLevelZoningAlgs}

\begin{algorithm}                      % enter the algorithm environment
\caption{False Positive Minimization}          % give the algorithm a caption
\label{alg:fpFiltering}                           % and a label for \ref{} commands later in the document
\begin{algorithmic}                    % enter the algorithmic environment
\STATE $aggregateData = []$
\STATE $windowStartIndex = 1$
\WHILE{$windowStartIndex < length(sensorFirings)$}
\STATE $windowEndTime = sensorFirings(windowStartIndex) + oneMin$
\STATE $firingsBeforeEnd = sensorFirings < windowEndTime$
\STATE $firingsInWindow = firingsBeforeEnd(windowStartindex:end)$
\IF{$length(firingsInWindow) \geq numSensors / 2$}
\STATE $aggregateData.append(firingsInWindow(end))$
\STATE $windowStartInterval = windowEndInterval + 1$
\ELSE
\STATE $windowStartInterval = windowStartInterval + 1$
\ENDIF
\ENDWHILE
\end{algorithmic}
\end{algorithm}

\begin{algorithm}                      % enter the algorithm environment
\caption{Search Space Generation}          % give the algorithm a caption
\label{alg:paramGen}                           % and a label for \ref{} commands later in the document
\begin{algorithmic}                    % enter the algorithmic environment
\FOR{$occStartCnt = occStartCntRange$}
\FOR{$occStartTO = occStartTORange$}
\FOR{$occEndCnt = occEndCntRange$}
\FOR{$occEndTO = occEndTORange$}
\STATE $occupancyTS = []$
\FOR{$i = 1:length(aggregateData)$}
\IF{$aggregateData(i + occStartCnt) - aggregateData(i) \leq occStartTO$}
\STATE $occupancyTS.append([aggregateData(i+occStartCnt-1), 1])$
\ELSIF{$aggregateData(i + occEndCnt) - aggregateData(i) > occEndTO$}
\STATE $occupancyTS.append([aggregateData(i) +
  occEndTO, 0])$
\ENDIF
\ENDFOR
\STATE $transCnt = 0$
\STATE $occDuration = 0$
\STATE $coveredFirings = []$
\STATE $iData = 1$
\STATE $vacancyStartTime = NaN$
\STATE $offIntervals = []$
\IF{$occupancyTS(1, 2) == 1$}
\STATE $occupancyStartTime = occupancyTS(1, 1)$
\ELSE
\STATE $occupancyStartTime = 0$
\ENDIF
\FOR{$j = 2:length(occupancyTS)$}
\IF{$occupancyTS(j - 1, 2) == 0 \: \AND \: occupancyTS(j, 2) == 1$}
\STATE $occupancyStartTime = occupancyTS(j, 1)$
\STATE $transCnt += 1$
\IF{$vacancyStartTime \neq NaN$}
\STATE $offIntervals.append(occupancyStartTime - vacancyStartTime)$
\ENDIF
\ELSIF {$occupancyTS(j - 1, 2) == 1 \: \AND \: occupancyTS(j, 2) == 0$}
\STATE $transCnt += 1$
\STATE $occDuration = occDuration +
occupancyTS(j, 1) - occupancyStartTime$
\WHILE{$aggregateData(iData) \leq occupancyTS(j,
  1)$}
\IF{$aggregateData(iData) \geq occupancyStartTime \: \AND \: aggregateData(iData) < occupancyTS(j, 1)$}
\STATE
$coveredFirings.append(aggregateData(iData))$
\ENDIF
\ENDWHILE
\ENDIF
\STATE $FNCnt = length(aggregateData) - length(coveredFirings)$
\STATE $offInt25 = prctile(offIntervals, 25)$
\STATE $params.append([FNCnt, transCnt, offInt25, occDuration, ...$
\STATE $occStartCnt, occStartTO, occEndCnt, occEndTO])$
\ENDFOR
\ENDFOR
\ENDFOR
\ENDFOR
\ENDFOR
\end{algorithmic}
\end{algorithm}

\begin{algorithm}                      % enter the algorithm environment
\caption{Parameter Selection}          % give the algorithm a caption
\label{alg:paramSelection}                           % and a label for \ref{} commands later in the document
\begin{algorithmic}                    % enter the algorithmic environment
\FOR{$i = 1:length(params)$}
\IF{$params(i, 1) / numDays < FN \: \AND \: params(i, 2) / numDays < K
  \: \AND \: params(i, 3) \geq T_{25}$}
\STATE $candidateParameters.append(params(i, :))$
\ENDIF 
\ENDFOR
\STATE $minOccupiedDurationIndex = find(min(candidateParameters(:, 4)))$
\STATE $selectedParameters.append(candidateParameters(minOccupiedDurationIndex, 5:end))$
\end{algorithmic}
\end{algorithm}


\begin{algorithm}                      % enter the algorithm environment
\caption{Temperature Estimation}          % give the algorithm a caption
\label{alg:temperatureEstimation}% and a label for \ref{} commands later in the document
\begin{algorithmic}                    % enter the algorithmic environment
\STATE $wholeHouseTempSum = 0$
\STATE $numRooms = 0$
\FOR{$room in rooms$}
\STATE $wholeHouseTempSum += room.temperature$
\STATE $numRooms += 1$
\ENDFOR
\STATE $wholeHouseAvg = wholeHouseTempSum / numRooms$

\STATE $conditionedRoomsTempSum = 0$
\STATE $numConditionedRooms = 0$
\FOR{$room in rooms$}
\IF{$room.damperClosed == False \: \AND \: room.transitionallyOccupied ==
  False$}
\STATE $conditionedRoomsTempSum += room.temperature$
\STATE $numConditionedRooms += 1$
\ENDIF
\ENDFOR
\STATE $conditionedRoomsAvg = conditionedRoomsTempSum / numConditionedRooms$

\STATE $occupiedRoomsTempSum = 0$
\STATE $numOccupiedRooms = 0$
\FOR{$room in rooms$}
\IF{$room.stablyOccupied$}
\STATE $occupiedRoomsTempSum += room.temperature$
\STATE $numOccupiedRooms += 1$
\ENDIF
\ENDFOR
\STATE $occupiedRoomsAvg = occupiedRoomsTempSum / numOccupiedRooms$
\IF{$mode == 'cool'$}
\STATE $averageTemp = min(wholeHouseAvg, conditionedRoomsAvg,
occupiedRoomsAvg)$
\ELSE
\STATE $averageTemp = max(wholeHouseAvg, conditionedRoomsAvg,
occupiedRoomsAvg)$
\ENDIF
\end{algorithmic}
\end{algorithm}


\begin{algorithm}                      % enter the algorithm environment
\caption{HVAC Stage Selection}          % give the algorithm a caption
\label{alg:hvacStageSelection}% and a label for \ref{} commands later in the document
\begin{algorithmic}                    % enter the algorithmic environment
\IF{$numOccupiedRooms > 0$}
\STATE $tempDiff = averageTemp - setpoint$
\ELSE
\STATE $tempDiff = averageTemp - setback$
\ENDIF

\IF{$mode == 'heat'$}
\STATE $tempDiff = -tempDiff$
\ENDIF

\IF{$hvacStage == -1$}
\IF{$tempDiff > 1.5$}
\STATE $hvacStage = 2$
\ELSIF{$tempDiff > 0.5$}
\STATE $hvacStage = 1$
\ELSE
\STATE $hvacStage = 0$
\ENDIF
\ELSIF{$hvacStage == 0$}
\IF{$tempDiff > 1.7$}
\STATE $hvacStage = 2$
\ELSIF{$tempDiff > 0.7$}
\STATE $hvacStage = 1$
\ENDIF
\ELSIF{$hvacStage == 1$}
\IF{$tempDiff > 1.7$}
\STATE $hvacStage = 2$
\ELSIF{$tempDiff < -0.5$}
\STATE $hvacStage = 0$
\ENDIF
\ELSE
\IF{$tempDiff < -0.5$}
\STATE $hvacStage = 0$
\ELSIF{$tempDiff < 0.3$}
\STATE $hvacStage = 1$
\ENDIF
\ENDIF
\end{algorithmic}
\end{algorithm}


\begin{algorithm}                      % enter the algorithm environment
\caption{Dump Zone Selection}          % give the algorithm a caption
\label{alg:dumpZoneSelection}% and a label for \ref{} commands later in the document
\begin{algorithmic}                    % enter the algorithmic environment
\FOR{$room in rooms$}
\STATE $activeRooms = []$
\STATE $dumpCandidate = []$
\IF{$room.stageRequest > 0 \: or \: room.transitionallyOccupied$}
\STATE $activeRooms.append(room)$
\ELSE
\STATE $dumpCandidates.append(room)$
\ENDIF
\ENDFOR

\STATE $dumpSets = []$

\FOR{$dumpSet in combinations(dumpCandidates)$}

\STATE $airflow = calculateAirflow(activeRooms, dumpSet)$

\STATE $minFlowChange = INF$
\FOR{$room in dumpSet$}
\IF{$room.openAirflow < minflow$}
\STATE $minFlowChange = room.openAirflow - room.closedAirflow$
\ENDIF
\ENDFOR

\IF{$airflow > safeAirflow \: \AND \: minFlowChange > airflow - safeAirflow$}
\STATE $dumpSets.append(dumpSet)$
\ENDIF

\ENDFOR

\STATE $bestEnergy = -INF$; $bestChanges = 0$; $bestDumpZone = []$
\FOR{$dumpSet in dumpSets$}

\STATE $inactiveRooms = rooms$
\FOR{$room in activeRooms$}
\STATE $inactiveRooms.remove(room)$
\STATE $activeZoneEnergy += voltages(hvacStage, dumpSet, room)$
\ENDFOR

\STATE $damperChanges = 0$
\FOR{$room in dumpSet$}
\STATE $inactiveRooms.remove(room)$
\IF{$room.damperClosed$}
\STATE $damperChanges += 1$
\ENDIF
\ENDFOR

\FOR{$room in inactiveRooms$}
\IF{$room.damperClosed == False$}
\STATE $damperChanges += 1$
\ENDIF
\ENDFOR

\IF{$activeZoneEnergy - damperCost * damperChanges > bestEnergy -
  damperCost * bestChanges$}
\STATE $bestDumpZone = dumpSet$
\STATE $bestEnergy = activeZoneEnergy$
\STATE $bestDamperChanges = damperChanges$
\ENDIF

\ENDFOR

\STATE $dumpZone = []$
\FOR{$room in bestDumpZone$}
\STATE $dumpZone.append(room)$
\STATE $estimatedAirflow = calculateAirflow(activeRooms, dumpZone)$
\IF{$estimatedAirflow > safeAirflow$}
\STATE $return dumpZone$
\ENDIF
\ENDFOR

\end{algorithmic}
\end{algorithm}

\begin{algorithm}                      % enter the algorithm environment
\caption{Airflow Calculation}          % give the algorithm a caption
\label{alg:calculateAirflow}% and a label for \ref{} commands later in the document
\begin{algorithmic}                    % enter the algorithmic environment
\STATE $airflow = 0$
\FOR{$room in rooms$}
\IF{$room in activeRooms \: \OR \: dumpSet$}
\STATE $airflow += room.openAirflow$
\ELSE
\STATE $airflow += room.closedAirflow$
\ENDIF
\ENDFOR
\end{algorithmic}
\end{algorithm}

\begin{figure}
  \begin{macrolab}
RTS = RunTimeSystem();
weatherdirect = RTS.getMotes('type', 'weatherdirect');
tempSensors = SensorVector(weatherdirect, 'temperature');
x10 = RTS.getMotes('type', 'X10');
motionSensors = SensorVector(x10, 'motion');
motionSensorIDs = uint8({[8 1 9], [2 6], [10 5], [4], [7 11], [12 14], [3 15]});
tempSensorIDs = uint8({[1 3], [2], [6 7], [4 9 11], [12 13], [5 14], [8 10]}); 
damperIDs = uint8({[3 5], [1 2], [8 4], [10], [11 15], [12 14], [5 7]});
nightStart = [0 0 0 2 0 0];
nightEnd = [0 0 0 7 0 0];
curState = 'On'
every(60000)
  mode = dbRead('zoning', 'mode', 'latest')
  damperState = dbRead('zoning', 'damper', damperIDs) 
  motionVals = motionSensors.sense();
  tempVals =  tempSensors.sense();
  [stablyOccupied, transitionallyOccupied] = assessOccupancy(motionVals, motionSensorIDs);
  aTemp = calculateAverageTemperature(tempVals, stablyOccupied, transitionallyOccupied, damperState)
  if length(stablyOccupied) > 0 || length(transitionallyOccupied) > 0
    SP = dbRead('zoning', 'setpoint', 'latest');
  else
    SP = dbRead('zoning', 'setback', 'latest');
  if mode == 'heat'
    deltaTemp = SP - aTemp;
  else
    deltaTemp = aTemp - SP;
  curState = getNextState(curState, deltaTemp);
  if curState > 0:
    dumpRooms = selectDumpZone(curState, stablyOccupied, transitionallyOccupied);
    hvacActuate(mode, curState, [stablyOccupied dumpRooms]);
  else
    hvacActuate(mode, curState, stablyOccupied);
  end
end
  \end{macrolab}
  \smallskip
  \hrule width 1\columnwidth
  \caption{MacroLab implementation of room-level zoning.}
  \label{code:cs2}
\end{figure}
