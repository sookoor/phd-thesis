\chapter{Day/Night Zoning Controller}
\label{appendix:g}

\begin{figure}
  \begin{macrolab}
RTS = RunTimeSystem();
weatherdirect = RTS.getMotes('type', 'weatherdirect');
tempSensors = SensorVector(weatherdirect, 'temperature');
zones = uint8({[2 3 4], [0 1 5 6]}); % Day/Night zones
nightStart = [0 0 0 2 0 0];
nightEnd = [0 0 0 7 0 0];
every(60000)
  mode = RTS.dbRead('mode')
  setpoint = RTS.dbRead('setpoint')
  tempVals =  tempSensors.sense();
  curTime = clock;
  curNightStart = nightStart;
  curNightStart(1:3) = curNightStart(1:3) + curTime(1:3);
  curNightEnd = nightEnd;
  curNightEnd(1:3) = curNightEnd(1:3) + curTime(1:3);
  isDay = false;
  if datenum(curTime) <= datenum(curNightStart) && datenum(curTime) > datenum(curNightEnd)  
    curZoneTemps = tempVals(zones{1});
    isDay = true;
  else
    curZoneTemps = tempVals(zones{2});
  end
  zoneTemp = mean(curZoneTemps);
  if mode == 'heat'
    if (zoneTemp < setpoint && mode == 'heat') || (zoneTemp > setpoint && mode
    == 'cool') 
      RTS.actuate('hvac', 'on')
      if isDay 
         RTS.actuate('registers', zones{1})
      else
         RTS.actuate('registers', zones{2})
      end
    else
      RTS.actuate('hvac', 'off')
  end
end
  \end{macrolab}
  \smallskip
  \hrule width 1\columnwidth
  \caption{MacroLab implementation of day/night zoning.}
  \label{code:cs1}
\end{figure}
